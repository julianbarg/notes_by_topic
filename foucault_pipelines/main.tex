\documentclass[12pt, man, natbib]{apa6}
\usepackage[USenglish]{babel}
\usepackage{setspace}
\usepackage{hyperref}

\title{Control and Imagination: Takeaway from Foucault}
\shorttitle{Power and Imagination}
\author{Julian Barg\\barg.julian@gmail.com}
\affiliation{Ivey Business School}
\setcitestyle{authoryear, open={()},close={)},citesep={,},aysep=}


 \abstract{}


\begin{document}
	
	\maketitle
	
	\singlespacing
	
	\section{}
	A world without oil--would that be a utopia or dystopia? For some, oil is a relic of the past, something that has to be overcome, ideally to be replaced with local, clean, renewable sources of energy. For others, oil is a reliable source of energy, abundantly available, maybe not entirely clean\footnote{Except maybe in the Troll-in-chief's "clean coal" universe.} but controllable--and a world without oil would be less stable, more prone to turmoil. Utopia or dystopia--the scenario of a world without oil seems equally unlikely from both perspectives. In the universe of petroleum engineering, pipeline technology is a success story. Over the last 50 years, we have witnessed a complete automation of the area, starting with remote valves, onto more and more sophisticated Supervisory Control and Data Acquisition (\textit{SCADA}) systems, and finally \textit{smart pigs} that allow for the operator to evaluate the condition of the pipeline from the inside. Pipeline spills occur not because pipelines are inherently unsafe, but can usually be traced back to human error and/or a concatenation of unfortunate events. Advances in microprocessor technology over the last thirty years have allowed the technology behind pipeline safety to become more compact, more precise, and more user-friendly. But have pipelines become as safe as they should be?
	
	Despite the technological advancements, pipeline spills are still prevalent. Even prestige projects like the Keystone Pipeline do not live up to their promises. There exists a gap between the engineer's idea of the pipeline per se, and our empirical reality. When the technology was still unproven, its future potential would bridge the gap between idea and reality. Now that the technology is more established, we are due for an analysis of the issue from another angle. What level of safety can we expect from pipelines, and should we satisfied with its performance, or look into alternatives?
	
	If one was to consider pipeline spills as a problem that is to be solved, one might come up with four different ways of doing so. (1) Provide arguments as to why the current level of pipeline spills is acceptable. (2) Make efforts to reduce the number of oil spills to zero (or any other number that is deemed acceptable). (3) Move transportation of oil onto rail, ship, or street. (4) Replace fossil fuels (gradually or abruptly) with other sources of energy. However, in the meantime pipelines have achieved the comfortable status of \textit{essential infrastructure}. Pipelines also compare favorably in terms of safety to the current main competitor: onshore transportation via truck. Historically, transportation of oil even within the contiguous US was often accomplished by ship, until German U-boats began attacking American trading vessels, including tank ships, off the Atlantic Coast.
	
	Let's take on the viewpoint of a pipeline engineer. The engineer would be aware of technologies that have additional potential to reduce the number of oil spills. After all, we have a great understanding of pipelines, what pressure is generate under load, and how to evaluate the condition of pipelines. Option (4), to replace pipelines with renewable for the sake of energy, would appear as an excessive response. And option (3) would not be appealing either--after all, pipelines compare favorably to the other options in terms of safety. In fact, it is the relative performance that made them a popular choice during the second world war in the first place. So then the remaining options are to accept the current level of pipeline spills, or to make efforts to reduce the number of spills. Once we arrive at these two options, our consideration set is already very much constrained. If one then was to take these considerations into a process where we move from the year $t$ to the year $t+1$, it would affect ones interpretation of developments during that year. If during this time period we observe concerning spills, and one was to conduct an analysis thereof, one would conclude that insufficient efforts (2) by the operator are the reason for the spill and more efforts need to be made (because options 3 and for have already been removed from the consideration set).
	
	If a pipeline engineer is one end of the spectrum, the other end of the spectrum would be an anti-pipeline activist.\footnote{Of course, there are not in my backyard (NIMBY) activists that do not object to pipeline technology \textit{per se}, but take offense at the specific location of construction. At the actual \textit{end} of the spectrum, there would be a proponent of decarbonization (or at least far-fetching changes to how fossil fuels are transported, processed, and utilized).} The activist would most likely rely on another (no less valid) epistemology. Just like the engineer may have chosen his career after some personal engineering-related experience, maybe the activist might have witnessed the damage that a pipeline spill can do to a community, and the lack of fairness in the restitution process. Although the activist and the engineer are very different, in one regard the two actors are very similar: the activist, like the engineer, would rule out a number of options right away. In fact (4) a transition to 100\% renewable energy, might be the only option that the activist considers.
	
	There is one significant difference between the engineer and the activist though that is relevant for the issue of pipeline safety. That difference is not related to their intrinsic attributes, but rather related to their actual impact on pipeline operations in the US. Although North America has witnessed many anti-pipeline protests with iconic moments, especially from indigenous groups, these protests only had limited impact on pipeline projects. Individual protests may challenge the specific design of an individual pipeline, but rarely has the construction of an entire pipeline been prevented. Engineers are often primary witnesses who drive the approval process of new pipelines--and their imaginations and calculations of the pipeline safety to be in place become prime exhibits. For instance, bold assumptions with regard to the pipeline's safety features helped along the permit process of Keystone XL \citep{Stansbury2011}.
	
	At least in the current political environment, the ability of engineers to affect pipeline operations is far greater than that of activists. Since 2018, parts of the US have witnessed efforts to crack down on energy activism. Starting in Louisiana, trespassing on infrastructure, including pipelines, has been made a felony, with 22 other states adopting similar laws.\footnote{These laws to back to an initiative of the American Legislative Exchange Council, which "shops around" legislations such as these in lobbying efforts. See \url{https://www.propublica.org/article/how-louisiana-lawmakers-stop-residents-efforts-to-fight-big-oil-and-gas} and \url{https://www.icnl.org/usprotestlawtracker/}, accessed 2020-06-22.} Constrained imagination (on both sides), in conjunction with a very unequal ability to influence decision making on pipeline matters, results in a situation where the status quo is sustained. Claims of pipelines being "safe" or problems being effectively addressed are not effectively challenged in the arena where decisions are being made.\footnote{The exception that proves the rule being that the Obama administration halted some pipeline projects, which were later greenlit by the Trump administration.} 
	
	Under these conditions, one side effectively holds a "monopoly" on the imagination of how pipeline systems could evolve. That is reflected in (a) the permitting process, and (b) the responses to pipeline spills. In (a) the permitting process, questionable claims (such as those surrounding the Keystone XL permitting process) are held up. For instance, Enbridge can continue to claim a response time of less than 20 minutes between a spill and the shutoff of a line--although actual spills have proven that this claim is far removed from reality\citep{Stansbury2011}. Radical, sweeping changes, such as the installation of external spill sensors along the full length of pipelines, are not considered. Furthermore, (b) responses to pipeline spills are limited to fines and obligations to update safety. Spills do not lead to the (permanent) shutdown of lines. Which in turn implies that safety updates can resolve any kind of safety issues in pipelines. The possibility that an intrinsic problem (e.g., of management) could permanently disqualify a line from being operated is not considered.
	
	Since other alternatives are not considered in this process, the imagination is constrained. How could the pipeline system evolve in a less constrained environment? If pipeline safety was considered on a case-by-case basis, certain pipelines with poor track records should certainly be disqualified from being operated any further (thus bringing actual consequences to pipeline spill into play). If also it is not only the physical object of the pipeline that is being considered, the debate of specific pipelines could more often result in negative decisions. More sweeping technological changes could enter our imagination, such as the aforementioned comprehensive external spill detectors. This technology is typically not considered, because it is expensive. But not considering that technology creates a logical fallacy: the technology is too expensive, therefore other technology, which in theory is almost as reliable, has to make do; when actually the reaction should be--if the technology to make the pipeline safe is too expensive, maybe the pipeline should not exist.
	
\bibliography{bibliography}

\end{document}