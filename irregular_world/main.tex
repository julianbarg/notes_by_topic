\documentclass[12pt, man, natbib]{apa6}
\usepackage[USenglish]{babel}
\usepackage{setspace}
\usepackage{hyperref}

\title{Messy World: How to Escape Unreliable Sustainability Indicators}
\shorttitle{Messy World}
\author{Julian Barg\\barg.julian@gmail.com}
\affiliation{Ivey Business School}
\setcitestyle{authoryear, open={()},close={)},citesep={,},aysep=}


% \abstract{}


\begin{document}
	
	\maketitle
	
	\singlespacing
	
	\section{}
	
	What do plastic in our oceans, mass extinction of insects, and oil spills have in common? They are all environmental phenomena of the Anthropocene that are not the outcome of well-regulated commercial activity. Rather, it seems that these phenomena are noisy, out of our control, and in some instances their occurrence is utterly unpredictable. Much of our sustainability research (at least at business schools) focuses on the regular, the easily measurable, and the empirically uncontested. In particular, high-level ESG indicators have received a lot of attention over the last couple of years. However, environmental impacts are distributed unequally across space, time, and actors. Therefore, instead of focusing on coverage and sample size, depending on the outcome of interest it may be advisable for us to focus our attention on a very small subset of observations.
	
	Here are some empirical examples. (1) Individual pollution events led to over 40,000 sites in the US that have been designated as superfund sites. Exposed individuals suffer from sometimes severly reduced life expectancy. In some cases (such as Love Canal) the sites had to be closed off to the public altogether. (2) As much as 9\% of methane transported in US pipelines may be escaping into the atmosphere \citep{Tollefson2013}. While methane is comparatively "clean" as a fuel when it is burned, it is a very potent climate greenhouse gas. A few actors account for a majority of this spilled methane \citep{Tollefson2013}. (3) Oil spills data is already noisy, but in 2005 this noisiness reached new highs when two hurricane hit the gulf coast in a matter of weeks. Over 400 spills occurred, with the cumulative spill volume surpassing that of the Exxon Valdez oil spill \citep{Cruz2009}.\footnote{Little attention was paid to individual spills, so there is little information available on "what actually happened".} (4) Only 20 companies account for roughly one third of global carbon emissions.\footnote{\url{https://www.theguardian.com/environment/2019/oct/09/revealed-20-firms-third-carbon-emissions}, accessed 2020-06-25.}
	
	ESG indicators aggregate different scores to a higher level to obtain overall scores. In this process, individual metrics that matter can be lost \citep{Kotsantonis2019}. The work of many in our field is motivated by individual events--a shocking event in the news that we cannot let stand, an issue that we think does not receive enough attention, or just something that leaves us fascinated. To be clear, there is a reason why one would want to aggregate sustainability scores. When an organization accomplishes a good environmental footprint, we should not then ignore severe problems in other areas, such as injuries and fatalities on the factory floor. Thus, we construct new scores that we hope may capture some latent characteristic \citep{Eccles2019}. But on our quest to find the perfect sustainability indicator, we always run the risk that a large enough dataset becomes noise that drowns out the individual interesting event. Similarly, even in large datasets extreme outliers can drive results, which may be unwelcome. But these extreme outliers may be what motivates the research in the first place, when individual bad actors "spoil" a whole industries. In this case, these individual actors may be deserving of much more of our attention than other actors.
	
	In my case, I have selected a single type of pollution event--pipeline spills--because I believe that this specific type of pollution is important and deserves attention. Individual events, such as the Kalamazoo River, are very serious and frankly should not occur. A systemic analysis might drown these events out, for instance, because the issue that receives (and is deserving of) our attention when analyzing the petroleum industry as a whole is greenhouse gas emissions. But to focus on the issue of greenhouse gas emissions would mean losing out of sight the individual fates that are linked directly to pipeline spills. Pipeline spills do not aggregate well beyond the pipeline industry. How would we build an indicator that allows us to compare e.g., a pipeline operator with a chemical company or a service company, while still giving introspective on oil spills such as the one that affected the Kalamazoo River.
	
	But that does not mean that by narrowing our view to pipeline spills, we circumvent all problems associated with quantified sustainability. The DOT provides a comprehensive dataset on pipeline spills in the US. This dataset includes variables such as spill volume, recovery volume, number of injuries and deaths, and environmental reclamation necessitated. Even on the scale of a project as specific as pipeline spills, social construction still needs to happen for me to be able to make comparisons across individual spills. The following are three examples. (1) What matters? The volume of the pipeline spill, or how much of the spill remains in the environment? (2) In some cases, oil spills into containment areas and is easily retrieved. In other cases, oil is retrieved by removing the top soil in the affected area. The first case is better for the environment, but quantitatively speaking--without qualitative information on the spill--the two cases might look equal. (3) Highly Volatile Liquids (HVLs) just dissipate when exposed to the atmosphere--does that mean that HVL spills do not matter?

\bibliography{bibliography}

\end{document}