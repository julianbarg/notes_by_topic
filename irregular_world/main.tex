\documentclass[12pt, man, natbib]{apa6}
\usepackage[USenglish]{babel}
\usepackage{setspace}
\usepackage{hyperref}

\title{Messy World: How to Escape Unreliable Sustainability Indicators}
\shorttitle{Messy World}
\author{Julian Barg\\barg.julian@gmail.com}
\affiliation{Ivey Business School}
\setcitestyle{authoryear, open={()},close={)},citesep={,},aysep=}


% \abstract{}


\begin{document}
	
	\maketitle
	
	\singlespacing
	
	\section{}
	
	What do plastic in our oceans, mass extinction of insects, and oil spills have in common? They are all environmental phenomena of the Anthropocene that are not the outcome of well-regulated commercial activity. Rather, it seems that these phenomena are noisy, out of our control, and in some instances their occurrence is utterly unpredictable. Much of our sustainability research (at least at business schools) focuses on the regular, the easily measurable, and the empirically uncontested. In particular, high-level ESG indicators have received a lot of attention over the last couple of years. However, environmental impacts are distributed unequally across space, time, and actors. Therefore, instead of focusing on coverage and sample size, depending on the outcome of interest it may be advisable for us to focus our attention on a very small subset of observations.
	
	Here are some empirical examples. (1) Individual pollution events led to over 40,000 sites in the US that have been designated as superfund sites. Exposed individuals suffer from sometimes severly reduced life expectancy. In some cases (such as Love Canal) the sites had to be closed off to the public altogether. (2) As much as 9\% of methane transported in US pipelines may be escaping into the atmosphere \citep{Tollefson2013}. While methane is comparatively "clean" as a fuel when it is burned, it is a very potent climate greenhouse gas. A few actors account for a majority of this spilled methane \citep{Tollefson2013}. (3) Oil spills data is already noisy, but in 2005 this noisiness reached new highs when two hurricane hit the gulf coast in a matter of weeks. Over 400 spills occurred, with the cumulative spill volume surpassing that of the Exxon Valdez oil spill \citep{Cruz2009}.
	
	There is already a significant body of literature on sustainability indicators--and their shortcomings. 

\bibliography{bibliography}

\end{document}