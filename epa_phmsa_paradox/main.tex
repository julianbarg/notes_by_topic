\documentclass[12pt, man, natbib]{apa6}
\usepackage[USenglish]{babel}
\usepackage{setspace}
\usepackage{hyperref}

\title{The Non-coupled Environment: Environmental Disaster Response and  }
\shorttitle{Short Title}
\author{Julian Barg\\barg.julian@gmail.com}
\affiliation{Ivey Business School}
\setcitestyle{authoryear, open={()},close={)},citesep={,},aysep=}


% \abstract{}


\begin{document}
	
	\maketitle
	
	\singlespacing
	
	\section{}

	Pipeline spills often provide scarring images. Neither river nor marsh, bird or man is safe from oil slicks, fumes, fires or explosions. Symptoms of gasoline poisoning include e.g., vision loss, dizziness, extreme fatigue, and vomiting. Loss of consciousness can also occur--in 1999 an 18-yer-old suffocated after he lost consciousness and fell face-first into a creek as the result of a gasoline spill in Bellingham, Washington.\footnote{He was one of three victims of the incident.}. The haunting images and stories of pipeline spills stand in stark contrast to reports produced by the Environmental Protection Agency (EPA) and the Pipeline and Hazardous Materials Safety Administration (PHMSA). These official documents usually are not overtly concerned with the human or environmental tragedy that has taken place, but rather discuss the actions that the response crew has taken, and the results they have achieved.
	
	With regard to disaster response, there is a potential conflict between how these organizations' goals and how measure performance. Their mission is to protect the environment as well as human health.\footnote{\url{https://www.epa.gov/aboutepa/our-mission-and-what-we-do} \& \url{https://www.phmsa.dot.gov/about-phmsa/phmsas-mission}, accessed 2020-06-18}. The EPA's and PHMSA's activities include documentation of pipeline spills as well as spill response. For PHMSA, documenting spills is mostly a "passive" activity--the PHMSA compiles the data that is provided to the agency and makes it accessible to the public. Therefore, the PHMSA represents itself as an agency that engages in safety initiatives. The EPA on the other hand produces many documents on disaster response and ecological restoration. The annual report of the EPA emphasizes the outcomes of their remediation and enforcement efforts.\footnote{\url{https://www.epa.gov/sites/production/files/2020-03/documents/fy21-cj-13-program-performance.pdf}, accessed 2020-06-18. To highlight the significance or this decision, one could also expect the EPA to provide a "state of the environment"--or in other words emphasize the need for the organization to receive funding in the first place, rather than highlighting the organization's efficient use of money.}
	
	Organizations typically assess their performance on the basis of indicators that they (at least in theory) can influence \citep{March1963}. In the case of the EPA and the PHMSA this leads to a potential conflict. The organizations task is to address environmental pollution--but both the EPA and the PHMSA have a motivation to present to the regulator a positive picture of environmental remediation as evidence of their good performance. 
	
\bibliography{bibliography}

\end{document}